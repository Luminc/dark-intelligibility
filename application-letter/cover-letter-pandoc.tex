Rijksmuseum Amsterdam

Datum: 9/22/2025 Onderwerp: Sollicitatie Data-Engineer

Geachte mevrouw Scheltjens,

Met grote fascinatie las ik de vacature voor Data Engineer. Hoewel mijn
profiel niet één-op-één aansluit bij de zware infrastructuur-eisen, voel
ik een enorme connectie met de missie van uw afdeling: het ontsluiten
van collectiedata via innovatieve, interdisciplinaire oplossingen. Ik
schrijf u dan ook met het oog op de potentie die mijn unieke profiel kan
bieden.

Als ontwerper en content strateeg met een achtergrond aan de
kunstacademie, heb ik mijn carrière gewijd aan het vertalen van de
verhalen achter objecten. In mijn meest recente rol in de veilingwereld
heb ik dit echter gecombineerd met een diepe duik in technologie. Ik heb
op eigen initiatief Python-scripts ontwikkeld die, via API's, de
volledige workflow van marketing en contentcreatie automatiseren.

Mijn passie voor technologie en mijn geloof in de potentie van AI zijn
groot. In mijn vrije tijd ben ik bezig met het bouwen van een
accounting-applicatie met PostgreSQL en Docker, waarbij ik moderne tools
zoals LLM's gebruik om mijn leercurve exponentieel te versnellen. Dit
toont niet zozeer mijn huidige expertise, maar wel mijn vermogen en
drive om complexe, nieuwe technologieën snel eigen te maken.

Ik zie een toekomst waarin de rijke data van het Rijksmuseum niet alleen
wordt ontsloten, maar ook wordt verrijkt en geactiveerd door
AI-toepassingen, bijvoorbeeld voor contentgeneratie of wetenschappelijk
onderzoek. Op dat snijvlak -- tussen de collectie, de data en de
innovatieve (AI-)toepassingen -- ligt mijn kracht.

Ik ben ervan overtuigd dat mijn combinatie van domeinkennis, bewezen
technische aanleg en een strategische visie op AI van grote waarde kan
zijn voor uw team. Mocht er naast de pure Data Engineer-rol ruimte zijn
voor een meer hybride profiel, dan zou ik mijn motivatie zeer graag in
een gesprek toelichten.

Met vriendelijke groet,

Jeroen Kortekaas
