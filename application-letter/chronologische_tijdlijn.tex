\documentclass[10pt,a4paper]{article}

% Essential packages
\usepackage[utf8]{inputenc}
\usepackage[T1]{fontenc}
\usepackage{microtype}
\usepackage[a4paper,left=13mm, right=13mm, top=13mm, bottom=13mm]{geometry}
\usepackage{setspace}
\usepackage{enumitem}
\usepackage{changepage}
\usepackage{longtable}
\usepackage{array}
\usepackage{booktabs}

% Font setup - clean sans-serif to match InDesign
\usepackage{fontspec}
\setmainfont{Manrope Light}[Scale=1.0]
\setsansfont{Manrope}[Scale=1.0]

% Remove page numbers and headers
\pagestyle{empty}

% Typography settings - 15pt grid/line height
\setlength{\baselineskip}{15pt}
\setlength{\parindent}{0pt}
\setlength{\parskip}{15pt}

% Table settings
\setlength{\tabcolsep}{8pt}
\renewcommand{\arraystretch}{1.4}

\begin{document}

% Set 9pt body font size and 15pt line spacing
\fontsize{9}{15}\selectfont

% Sender branding at top - right aligned
\noindent
\hfill
\begin{minipage}[t]{0.5\textwidth}
\raggedleft
{\fontsize{11}{16}\selectfont Jeroen Kortekaas} \\
Maria Tesselschadelaan 45 \\
2135 RD Hoofddorp \\
The Netherlands \\
studio@jeroenkortekaas.com \\
\end{minipage}

\vspace{1.5em}

% Body content with 28mm left/right margins
\begin{adjustwidth}{15mm}{15mm}

% Document title and info
{\fontsize{13}{18}\selectfont\textbf{VERTROUWELIJK JURIDISCH DOSSIER}}

\vspace{1em}

\end{adjustwidth}

\begin{longtable}{>{\raggedright}p{3cm}>{\raggedright\arraybackslash}p{13.5cm}}
\textbf{Onderwerp} & Chronologisch Overzicht van de Werkrelatie en Geschil \\[0.8em]
\textbf{Betreft} & Jeroen Kortekaas (Opdrachtnemer) \\
& vs. Jurriën Kuijper / Uw Ontruimer B.V. / Vendulion (Opdrachtgever) \\[0.8em]
\textbf{Datum opgesteld} & 15 september 2025 \\[0.8em]
\textbf{Status} & Definitief, ter voorbereiding op juridische stappen \\
\end{longtable}

\begin{adjustwidth}{15mm}{15mm}

\vspace{1em}

{\fontsize{11}{16}\selectfont\textbf{EXECUTIVE SUMMARY}}

Dit dossier documenteert een tweejarige werkrelatie die evolueerde van een informele, ``zwarte'' constructie naar een juridisch precaire freelance-opzet. Vanaf mei 2025 ondernam Dhr. Kortekaas, op basis van een grondige analyse (\textbf{Bijlage C}), proactief stappen om de risico's van schijnzelfstandigheid voor beide partijen te mitigeren. Zijn voorstel voor een professionele arbeidsrelatie werd beantwoord met een onacceptabel payroll-voorstel.

Hierop stelde Dhr. Kortekaas een duidelijk en gedocumenteerd ultimatum: een direct dienstverband, of een verhoogd, risicodekkend freelance tarief van €24/uur (\textbf{Bijlage D}). De opdrachtgever was nalatig hierop te reageren, liet de werkzaamheden doorgaan en aanvaardde daarmee stilzwijgend de nieuwe voorwaarden. Toen de factuur voor juli conform dit tarief werd ingediend, probeerde de opdrachtgever achteraf tevergeefs te onderhandelen. Pas nadat Dhr. Kortekaas dreigde zijn werkzaamheden op te schorten, werd de juli-factuur volledig betaald, wat het tarief van €24/uur de facto valideerde. Om de betaling van de juli-factuur te bespoedigen en verdere escalatie te voorkomen, ging Dhr. Kortekaas onder druk akkoord met een eenmalige, tot augustus beperkte concessie. Het huidige geschil ontstond toen de opdrachtgever weigerde de facturen voor augustus (met concessie) en september te betalen. Zijn bewering dat cruciale discussies niet hebben plaatsgevonden, wordt direct weerlegd door zijn eigen schriftelijke communicatie (\textbf{Bijlage A1}), wat zijn geloofwaardigheid ernstig ondermijnt.

De beëindiging van de samenwerking en het huidige geschil over de openstaande facturen (€4.197,64 incl. btw) zijn een direct gevolg van de onwil van de opdrachtgever om een professionele en wettelijk correcte arbeidsrelatie aan te gaan. Dit gedrag vindt plaats tegen een achtergrond van door Dhr. Kortekaas gedocumenteerde, systematische bedrijfsfraude (shill bidding, belastingontduiking), wat een patroon van onethische en onwettige bedrijfsvoering aantoont en zijn geloofwaardigheid in dit geschil ernstig ondermijnt.

\vspace{1em}

{\fontsize{11}{16}\selectfont\textbf{LIJST VAN BETROKKEN PARTIJEN}}

\begin{enumerate}
\item \textbf{Jeroen Kortekaas} (Opdrachtnemar)
\item \textbf{Jurriën Kuijper} (Opdrachtgever, handelend onder Uw Ontruimer B.V. / Vendulion)
\item \textbf{Alfredo, Manauris, Dioser et al} (Andere medewerkers, relevant als bewijs voor bedrijfscultuur en personeelsbeleid)
\end{enumerate}

\vspace{1em}

{\fontsize{11}{16}\selectfont\textbf{LIJST VAN BEWIJSSTUKKEN}}

\end{adjustwidth}

\begin{longtable}{c>{\raggedright}p{6cm}>{\raggedright\arraybackslash}p{8.5cm}}
\textbf{Ref.} & \textbf{Document} & \textbf{Relevantie} \\[1em]
\textbf{A} & \texttt{WhatsApp Chat with Jurriën.txt} & Originele WhatsApp communicatie (26 maanden). \\[0.5em]
\textbf{A1} & \texttt{Emails en Chatgeschiedenis/Werknemerrelatie\_WhatsApp.md} & Analyse van WhatsApp communicatie met citaten en verwijzingen naar bijlagen. \\[0.5em]
\textbf{B} & \texttt{Payroll\_contract.pdf} & Bewijs van het onredelijke payroll-voorstel van 4 juni 2025. \\[0.5em]
\textbf{C} & \texttt{voorbereiding\_mei.md} & Bewijs van de proactieve en inhoudelijke voorbereiding op het gesprek over schijnzelfstandigheid. \\[0.5em]
\textbf{D} & \texttt{Emails en Chatgeschiedenis/Gmail - Reactie op voorstel.pdf} & Email over het ultimatum als reactie op pay-roll voorstel met tegenvoorstel. \\[0.5em]
\textbf{F} & \texttt{Emails en Chatgeschiedenis/Gmail - Factuur 2025-017 Studio Jeroen Kortekaas.pdf} & Email-betwisting van factuur door opdrachtgever. \\[0.5em]
\textbf{G} & \texttt{Emails en Chatgeschiedenis/Gmail - Herinnering Factuur 2025-017 Studio Jeroen Kortekaas.pdf} & Herinnering en verdere correspondentie over betwiste factuur. \\[0.5em]
\textbf{H} & \texttt{Facturen/2025-016\_Vendulion\_JeroenKortekaas.pdf} & De factuur voor \textbf{juli}, gebaseerd op €24/uur, die eind augustus is betaald. \\[0.5em]
\textbf{I} & \texttt{Facturen/2025-017\_Vendulion\_JeroenKortekaas.pdf} & Factuur voor \textbf{augustus}, gebaseerd op eenmalige concessei van €22,50/uur (onderdeel van het geschil). \\[0.5em]
\textbf{J} & \texttt{Facturen/2025-018\_Vendulion\_JeroenKortekaas.pdf} & Aanvullende factuur (onderdeel van het geschil). \\[0.5em]
\textbf{K} & \texttt{Fraude\_Analyse\_Vendulion\_Veilingpraktijken.md} & Gedetailleerd bewijs van shill bidding en belastingfraude. \\[0.5em]
\textbf{L} & \texttt{UrenspecificatieNovember2023\_uurtarief12\_24.jpg} & Bewijs van uurtarief in informele periode. \\[0.5em]
\textbf{M} & \texttt{UrenspecificatieApril2024\_uurtarief12\_24.jpg} & Bewijs van overgang naar officiële freelance-status. \\[0.5em]
\textbf{N} & \texttt{Vendulion Urenregistratie.xlsx} & \textbf{COMPLETE urenregistratie juni 2023 - september 2025 (inclusief zwarte periode).} \\[0.5em]
\textbf{O} & \texttt{Overdrachtsnotitie.md} & \textbf{Bewijs van gespecialiseerd werk:} AI-automatisering, Python scripting, complex systeem management - toont essentiële bedrijfsfunctie. \\[0.5em]
\textbf{Q} & \texttt{Emails en Chatgeschiedenis/Gmail - Formele Ingebrekestelling en Laatste Aanmaning - Facturen 2025-017 \& 2025-018 Studio Jeroen Kortekaas.pdf} & Formele ingebrekestelling en laatste aanmaning met definitieve betalingstermijn tot 1 oktober 2025. \\
\end{longtable}

\begin{adjustwidth}{15mm}{15mm}

{\fontsize{10}{15}\selectfont\textbf{AANVULLENDE BEWIJSSTUKKEN - VOLLEDIGE FACTURENSET}}

\end{adjustwidth}

\begin{longtable}{c>{\raggedright}p{4.5cm}>{\raggedright}p{3.5cm}>{\raggedright\arraybackslash}p{6cm}}
\textbf{Ref.} & \textbf{Document} & \textbf{Periode} & \textbf{Relevantie} \\[1em]
\textbf{P1-P18} & \texttt{Facturen/2024-004 t/m 2025-018} & April 2024 - September 2025 & Volledige geschiedenis freelance-relatie met 18 facturen (consistent uurtarief patroon). \\
\end{longtable}

\begin{adjustwidth}{15mm}{15mm}

\vspace{1em}

{\fontsize{11}{16}\selectfont\textbf{GEDETAILLEERDE CHRONOLOGISCHE TIJDLIJN}}

{\fontsize{10}{15}\selectfont\textbf{Fase 1: Informele Werkrelatie (Juni 2023 – April 2024)}}

\begin{itemize}
\item \textbf{29 juni 2023:} Start van de werkzaamheden op informele basis (``zwart'') tegen een tarief van €12,24 per uur.
\item \textbf{Kenmerken:} Vanaf het begin is er een duidelijke gezagsverhouding, kenmerkend voor een dienstverband. De opdrachtgever bepaalt de taken, tijden en locaties.
    \begin{itemize}
    \item \textbf{Directe Instructies:} ``Kan jij morgen naar de loods in Amstelveen komen? 09:00'' (7 juli 2023).
    \item \textbf{Eenzijdige Planningswijzigingen:} ``Morgen loopt allemaal weer iets uit haha. 10:00 morgen'' (2 juli 2023).
    \item \textbf{Vast Patroon:} Dhr. Kortekaas werkt op vaste dagen (maandag, donderdag, vrijdag), wat duidt op structurele inbedding in de organisatie.
    \item \textbf{Verlof en Ziekte:} Verzoeken om vrije dagen en ziekmeldingen worden gecommuniceerd zoals een werknemer dat zou doen. ``Hey Jurriën, ik zit ziek thuis vanwege een oorontsteking met koorts.'' (16 augustus 2023).
    \end{itemize}
\end{itemize}

{\fontsize{10}{15}\selectfont\textbf{Fase 2: Officiële Freelance Relatie \& De Verhullingsfase (April 2024 – Mei 2025)}}

\begin{itemize}
\item \textbf{April 2024:} Overgang naar een officiële freelance-status tegen een tarief van €16,53 excl. btw. De aard van het werk en de gezagsverhouding blijven onveranderd.
\item \textbf{Januari 2025:} Op verzoek van de opdrachtgever stopt Dhr. Kortekaas met het specificeren van uren op facturen. Dit was een cosmetische aanpassing om de structurele, uren-gebaseerde relatie te verhullen en \textbf{geen} overgang naar een projectbasis. De gefactureerde bedragen bleven consistent met het uurtarief en het gemiddelde aantal gewerkte uren.
    \begin{itemize}
    \item \textbf{Bewijs van voortdurende urenregistratie:} De WhatsApp-geschiedenis toont aan dat de opdrachtgever zélf in deze periode nog expliciet om een urenspecificatie vroeg (o.a. op 9 april 2025: ``Wil jij nog je uren spec sturen over whatsapp''), wat de bewering van een 'projectbasis' direct ontkracht (\textbf{Bijlage A1}).
    \item \textbf{Voortdurende Instructies:} De gezagsverhouding blijft identiek. ``Pas jij de MP advertenties aan'' (24 juni 2025). ``Wil je de man bellen, dat deze morgenochtend bezorgd wordt...'' (19 juni 2025).
    \end{itemize}
\end{itemize}

{\fontsize{10}{15}\selectfont\textbf{Fase 3: Poging tot Professionalisering \& Escalatie (Mei – Juni 2025)}}

\begin{itemize}
\item \textbf{Mei 2025:} Dhr. Kortekaas bereidt zich grondig voor op een gesprek om de onhoudbare situatie van schijnzelfstandigheid aan te kaarten. Hij analyseert de risico's voor beide partijen en stelt professionele oplossingen voor (\textbf{Bijlage C}).
\item \textbf{30 mei 2025:} Na een gesprek waarin de opdrachtgever aangeeft met een contract bezig te zijn, stelt Dhr. Kortekaas proactief per WhatsApp voor om als overbrugging een tarief van \textbf{€24,00 excl. btw} te hanteren. Dit is de eerste keer dat dit specifieke, risicodekkende tarief wordt voorgesteld.
\item \textbf{4 juni 2025:} Als reactie biedt de opdrachtgever een payrollovereenkomst aan via ``Please'' (\textbf{Bijlage B}). Dit voorstel is een \textbf{nul-urencontract} tegen een effectief lager all-in tarief (€16,45) en wordt door Dhr. Kortekaas terecht als een verslechtering afgewezen.
\item \textbf{Begin juni 2025:} Na afwijzing van het payroll-voorstel, en geconfronteerd met de ontkenning en nalatigheid van de opdrachtgever om tot een serieuze overeenkomst te komen, formaliseert Dhr. Kortekaas zijn positie. Hij stuurt een ultimatum per e-mail (\textbf{Bijlage D}) waarin hij de reeds besproken voorwaarden voor voortzetting van de werkzaamheden na juni herhaalt:
    \begin{enumerate}
    \item \textbf{Voorkeur:} Een direct dienstverband.
    \item \textbf{Alternatief:} Een freelance tarief van \textbf{€24,00 excl. btw} en correcte, gespecificeerde facturatie.
    \end{enumerate}
\end{itemize}

{\fontsize{10}{15}\selectfont\textbf{Fase 4: Stilzwijgende Acceptatie \& Conflict (Juli – Augustus 2025)}}

\begin{itemize}
\item \textbf{Juli 2025:} De opdrachtgever reageert niet inhoudelijk op het ultimatum, maar laat de werkzaamheden doorgaan. Hiermee aanvaardt hij stilzwijgend de enige overgebleven voorwaarde waaronder Dhr. Kortekaas bereid was te werken: het tarief van €24/uur.
\item \textbf{1 augustus 2025:} Dhr. Kortekaas factureert juli conform de nieuwe voorwaarden.
\item \textbf{19 augustus 2025:} Ruim na het verstrijken van de betalingstermijn betwist de opdrachtgever de factuur. Pas nadat Dhr. Kortekaas dreigt zijn werkzaamheden per direct op te schorten, staakt de opdrachtgever de discussie over de juli-factuur. Om de betaling te bespoedigen en de samenwerking op dat moment niet volledig te laten escaleren, gaat Dhr. Kortekaas onder druk akkoord met een eenmalige en expliciet tot augustus beperkte concessie van €22,50/uur, die op de toekomstige augustus-factuur zal worden toegepast.
\item \textbf{Eind augustus 2025:} Ondanks de eerdere verbale betwisting, betaalt de opdrachtgever de volledige factuur voor juli (2025-016). Deze betaling valideert de facto het tarief van €24/uur voor de werkzaamheden in die maand.
\end{itemize}

{\fontsize{10}{15}\selectfont\textbf{Fase 5: Definitieve Breuk (September 2025)}}

\begin{itemize}
\item \textbf{2 september 2025:} Dhr. Kortekaas stuurt de factuur voor augustus (2025-017) met het concessietarief van €22,50/uur, maar bevestigt dat september tegen €24/uur gefactureerd zal worden.
\item \textbf{9 september 2025:} Tijdens een gesprek betwist de opdrachtgever de gemaakte afspraken over het tarief. Wanneer Dhr. Kortekaas vraagt wat hij dan heeft geschreven, antwoordt de opdrachtgever ontwijkend ``Dat weet je zelf ook wel''. De opdrachtgever probeert vervolgens een lager ``marktconform'' salaris voor te stellen (``Mijn boekhouder zegt...''), waarop Dhr. Kortekaas reageert dat hij daarvoor iemand moet zoeken. De opdrachtgever stemt hiermee in (``Ok''), waarmee de samenwerking eindigt. Dhr. Kortekaas maakt direct gebruik van zijn opschortingsrecht en behoudt de bedrijfssleutel en overdrachtsdocumentatie (Inclusief zelf geschreven automatisering software en digitale integraties) tot volledige betaling van openstaande facturen.
\item \textbf{10 september 2025:} Dhr. Kortekaas stuurt de factuur voor september (2025-018) voor werkzaamheden 1-9 september tegen het afgesproken tarief van €24/uur.
\item \textbf{15 september 2025:} De opdrachtgever betwist beide facturen en probeert Dhr. Kortekaas terug te dwingen naar de verhullende ``projectbasis'' facturatie, terwijl hij de gevoerde discussies over de werkrelatie ontkent (\textbf{Bijlage F, G}).
\item \textbf{15 september 2025:} Dhr. Kortekaas weigert definitief de facturen aan te passen en maakt gebruik van zijn opschortingsrecht door de bedrijfssleutel en overdrachtsdocumentatie te behouden tot volledige betaling heeft plaatsgevonden.
\item \textbf{17 september 2025:} In een e-mailreactie dreigt de opdrachtgever onrechtmatig de kosten voor het vervangen van de sloten te verhalen op Dhr. Kortekaas. Dit is een directe reactie op het rechtmatig toepassen van het opschortingsrecht en kan worden gezien als een poging tot intimidatie.
\item \textbf{24 september 2025:} De opdrachtgever stuurt een reeks intimiderende WhatsApp-berichten. Hij beweert ten onrechte dat een incassobureau niet kan worden ingeschakeld ``zolang we hier een geschil hebben, ik spreek uit ervaring.'' Dit is een poging om Dhr. Kortekaas te ontmoedigen juridische stappen te ondernemen. Verder meldt hij dat de verhuurder opdracht heeft gekregen de sloten te vervangen, wat een verdere escalatie is en een directe ondermijning van het opschortingsrecht van Dhr. Kortekaas.
\end{itemize}

\vspace{1em}

{\fontsize{11}{16}\selectfont\textbf{ANALYSE EN JURIDISCHE CONCLUSIE}}

\begin{enumerate}
\item \textbf{Er was sprake van een Arbeidsovereenkomst:} De werkrelatie voldeed gedurende de gehele periode aan de materiële criteria van een arbeidsovereenkomst (loon, arbeid, gezagsverhouding). De freelance-constructie was schijnzelfstandigheid. De aanwezigheid van andere medewerkers in een vergelijkbare positie (Alfredo, Menauris, Dioser et al) toont een structureel patroon binnen de bedrijfsvoering aan.

\item \textbf{De Vordering is Rechtsgeldig en Aantoonbaar Geaccepteerd:}
De rechtsgeldigheid van de vordering rust op twee pijlers:
Aanvaarding door Uitvoering: Het tarief van €24/uur, vastgelegd in het ultimatum van juni, werd door de opdrachtgever stilzwijgend aanvaard door de werkzaamheden in juli te laten voortduren.
Aanvaarding door Betaling: Nadat de opdrachtgever de juli-factuur betwistte, heeft hij deze uiteindelijk alsnog volledig betaald. Deze betaling, hoewel afgedwongen nadat Dhr. Kortekaas dreigde zijn werkzaamheden op te schorten, vormt een feitelijke en juridische validatie van het tarief van €24/uur voor die periode.
De latere poging van de opdrachtgever om voor de daaropvolgende maanden terug te vallen op een onjuiste ``projectbasis'' is dan ook niet alleen een manipulatieve actie, maar ook in strijd met zijn eigen eerdere handelen.

\item \textbf{Context van Onethische Bedrijfsvoering:} De weigering om een legitieme factuur te betalen en de pogingen tot manipulatie moeten worden gezien in de context van de algehele bedrijfsvoering. De forensische analyse (\textbf{Bijlage K}) toont een patroon van opzettelijke en systematische fraude (schijnbiedingen en belastingontduiking). Dit gedragspatroon vestigt het beeld van een opdrachtgever die bereid is wettelijke en ethische grenzen te overschrijden voor financieel gewin. Deze context is van materieel belang voor de beoordeling van zijn geloofwaardigheid en handelswijze in het huidige geschil.
\end{enumerate}

\end{adjustwidth}

\end{document}